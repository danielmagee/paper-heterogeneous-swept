\section{Implementation} \label{sec:hImplement}

\subsection{Primary data structure} \label{sec:hPrimaryData}

Implementing the swept rule for problems amenable to single-step PDE schemes is straightforward,
but dealing with more realistic problems often requires more complex, multi-step numerical schemes.
Managing the working array and developing a common interface for these schemes, provokes design decisions that have substantial impacts on performance.

One strategy for dealing with this complexity we term \texttt{flattening} since it flattens the domain of dependence in the time dimension by using wider stencils and fewer sub-timesteps. 
This strategy is more memory efficient for the working array which contains instances of the primary data structure at each spatial point, but it cannot easily accommodate different methods and equations.
It also introduces additional complexity from parsing the arrays and requires additional register memory for function and kernel arguments and ancillary variables.

In the new implementation shown here we use the \texttt{lengthening} strategy, also referred to as ``atomic decomposition'', which is instantiated as a struct to generalize the stages into a user-defined data type~\cite{WangDecomp}.
It requires more memory to be used in the primary data structure; for instance, our \texttt{flattening} version
of the Euler equations carried six doubles per spatial point since the pressure ratio used by the
limiter was rolled into the flattened step.
These strategies are described in Figures~\ref{f:Longpseudo} and ~\ref{f:flatPseudo}.
By restricting the stencil to three points, the \texttt{lengthening} method requires the pressure ratio to be stored and passed through the memory hierarchy meaning the data structure carries seven doubles per spatial point for the Euler equation.

\begin{figure}[bt]
    \begin{minipage}[t]{0.48\textwidth}
        \begin{lstlisting}
        // Q = {rho, rho*u, rho*E}
        struct states {
            double3 Q[2]; // State Variables
            double Pr; // Pressure ratio
        };

        __device__ __host__
        void stepUpdate(states *state, const int idx, const int tstep)
        {
            int ts = tstep % 4; // 4 is number of steps in cycle
            if (tstep & 1)  pressureRatio(state, idx, ts);
            else            eulerStep(state, idx, ts);
        }

        __global__ void classicStep(states *state, const int tstep)
        {
            int gid = blockDim.x * blockIdx.x + threadIdx.x + 1;
            stepUpdate(state, gid, tstep);
        }
        \end{lstlisting}
        \caption{Skeleton for the \texttt{lengthening} method in the \texttt{Classic} program. The \texttt{states} structure contains all the information to step forward at any point.  The user is only responsible for writing the \texttt{eulerStep} and \texttt{pressureRatio} functions and accessing the correct members based on the timestep count}
        \label{f:Longpseudo}
    \end{minipage}
    ~
    \begin{minipage}[t]{0.48\textwidth}
        \begin{lstlisting}
        __global__ void classicStep(const double *s_in, double *s_out, bool final)
        {
        int gid = blockDim.x * blockIdx.x + threadIdx.x;
        //number of spatial points - 1
        int lastidx = ((blockDim.x*gridDim.x));
        int gids[5];

        for (int k = -2; k<3; k++) gids[k+2] = (gid + k) % lastidx;

        //Final is false for predictor step, true otherwise.
        if (final) s_out[gid] += finalStep(s_in, gids);
        else       s_out[gid]  = predictorStep(s_in, gids);
        }
        \end{lstlisting}
        \caption{Skeleton for the \texttt{flattening} method in the \texttt{Classic} program. The sub-timesteps are compressed to a step with a wider stencil. The two arrays which alternate reading and writing are explicitly passed and traded in the calling function.}
        \label{f:flatPseudo}
    \end{minipage}
\end{figure}

To gauge the influence of this change in primary data structure, we implemented each combination of the classic and swept decomposition techniques using the \texttt{flattening} and \texttt{lengthening} data structures.
We applied these solvers to the Euler equations using the conditions and hardware described in Section~\ref{sec:ExpMethod}.
These programs use a single GPU as the sole computational device as opposed to the other results in this paper, which are evaluated with CPUs and GPUs on a heterogeneous platform.  

\hfigs{SpeedupAtomicArrayEuler}
{SpeedupAlgoAtomicArrayEuler}
{Speedup of \texttt{flattening} compared to the same scheme using \texttt{lengthening}.}
{Speedup of \texttt{Swept} program compared to \texttt{Classic} using the \texttt{flattening} and \texttt{lengthening} strategies}
{Performance comparison of the \texttt{flattening} and \texttt{lengthening} strategies using the Euler equations on the GPU only.}

% \begin{figure}[htbp]
%     \centering
%     \includegraphics[width=.7\textwidth]{SpeedupAtomicArray}
%     \caption{Speedup of \texttt{flattening} compared to the same scheme using \texttt{lengthening} applied to the KS equation on the GPU only.}
%     \label{f:lflat}
% \end{figure}

Figure~\ref{f:SpeedupAtomicArrayEulerResult} compares the performance of the memory storage techniques in an
experiment executed on a workstation with an \WCPU{} and an \WGPU{}.
The \texttt{Classic} program using the \texttt{flattening} method using is compared to the same program using the \texttt{lengthening} method and likewise for \texttt{Swept}. %----%
As Figure~\ref{f:SpeedupAtomicArrayEuler} shows, the \texttt{flattening} method is faster than \texttt{lengthening} for both decomposition methods and gets faster as the number of spatial points increases.
There is not much difference in the trends between swept and classic decomposition methods, but it does appear that the less performative data structure affects the \texttt{Swept} program more.
Much of the reason that \texttt{lengthening} is substantially less performative is an artifact of GPU architecture, so it is difficult to generalize the results of this study to heterogeneous systems. 
GPU memory accesses, from global and shared memory, are more sensitive to irregularities because of the parallel nature of the device as a whole; it is not designed to be used serially.
The extra memory requirements of the \texttt{lengthening} method also consume limited shared memory resources on the GPU, which diminishes the L1 cache capacity used to accelerate global memory accesses on Kepler-generation GPUs.
While locality is a significant issue for effective CPU memory accesses, it has a larger impact on GPU performance.

We feel that it is important to present these findings so that others who implement the swept rule will have a more thorough understanding of the tradeoffs inherent in the program design choices across architectures.
But, since the \texttt{Swept} and \texttt{Classic} algorithms perform similarly compared to the \texttt{flattening} method, the conclusions we derive from our experiments using the \texttt{lengthening} method remain valid.
And we believe that, in this case, the values that the \texttt{lengthening} method provides: extensibility and regularity are of greater value than the absolute best performance.

\subsection{Program design features}

Our earlier implementation of the swept rule for GPUs~\cite{OurJCP} required creating separate single-file
\CC{}\slash CUDA programs for each problem.
We have determined that this approach is insufficient for enabling the support for further exploration that is essential to the accurate analysis of highly architecture
and problem dependent algorithms.
Our previous analysis showed that the performance characteristics of the GPU-based swept rule depend
on the boundary conditions, numerical method, and governing equation(s).
In this work we endeavor to create a convenient and reusable interface through which other can reproduce our experiments, explore different equations and numerical methods to examine performance characteristics the performance characteristics of the swept rule.

We have created this interface by separating the domain decomposition and grid generation, and passing
a map between the equation-specific part of the code, defined by the user, and the generic part of the code.
In practice, the user defines an initialization procedure that defines any constant terms in the
governing equation(s), e.g., the Fourier number in the heat equation.
The user is guaranteed to receive the standard grid variables, e.g., $\Delta x$, $\Delta t$, in the map
along with any other values needed to define the constant terms, i.e., the thermal diffusivity
for the heat equation.
The user defines these non-standard constant terms in a JSON file passed to the program on the command line
or as command line key/value pair arguments. By defining these fundamental concepts in a separate JSON
file, variables that define the grid or material constants in the equations can be redefined without
the need to recompile.

This structure requires a solver interface that operates on every equation and numerical scheme
the same way, which in turn requires the use of a different strategy for decomposing more complex
equations and domains.
The clearest way to create a general form for a user-defined equation is to oblige the user to decompose
the numerical formula into atomic stages, a series of steps requiring only three-point stencils as described
by Wang~\cite{WangDecomp}, and provide a step counter in the root function to define the sequence of stages.
The user must also define the data structure where results of each stage are collected, this structure takes the form of a plain old data struct named \texttt{states}.

Our first study showed that shared memory is the most effective means of exploiting the memory hierarchy
for this application on this generation of GPUs~\cite{OurJCP}.
We rely on this conclusion here to again use the GPU characteristics to determine the size of the
domains of dependence.
Each GPU thread is mapped to a single spatial point and each CPU process, having only one available thread,
traverses a number of domains in serial.
This limits the size of the domain of dependence to the allowable number of threads in each block launched
in the GPU kernel, which depends on the occupancy of the most-restrictive kernel as determined by the shared
memory and register resource requirements.


Our approach organizes the available processors by assigning each GPU on a node to one
MPI process (i.e., CPU core) on the same node that has exclusive control over it; that process also
manages one domain on either side of the GPU subdomain.
Thus, to facilitate this, each MPI process comprises a positive, even number of subdomains.
All subdomains on a node must contain the same number of points, and all MPI processes evaluate the
same number of subdomains.
Processes that control a GPU simply ``contain'' additional subdomains equal to the GPU affinity times
the number of domains assigned to an MPI process.
For example, a domain of 160 points could be decomposed into subdomains of 16 points on a node with
four CPU cores and one GPU; each processor would compute the time-stepping for two subdomains, while the
MPI process controlling the GPU comprises four subdomains in total (two on the CPU, and two on the GPU).
This corresponds to a GPU affinity of one, since the GPU subdomain equals the CPU subdomain sizes.

Assigning a GPU to a single process reduces complexity and avoids using the GPU at the spatial
boundaries where imposing boundary conditions causes thread divergence.
This is useful from a conceptual standpoint, even though we previously found that boundary
conditions only affect GPU performance in a minor way~\cite{OurJCP}.

Our program uses the MPI\allowbreak+CUDA paradigm, and assigns one MPI process to each core for the life of a program.
We considered using an MPI\allowbreak+OMP\allowbreak+CUDA paradigm by assigning an MPI process to each socket, and
launching threads from each process to occupy the individual cores, but recent work has shown that
this approach rarely improves performance on clusters of limited size for finite volume or finite
difference solvers~\cite{IDAHO_MPI_CUDA, PerfAnalysisHetero}.
This conclusion has led widely used libraries, such as PETSc, to opt against a paradigm of threading
within processes~\cite{MillsPetsc}.
