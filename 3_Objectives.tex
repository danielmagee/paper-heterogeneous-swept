\section{Objectives} \label{sec:obj1}

This study concerns the construction and analysis of programs that apply swept time-space decomposition to explicit stencil computations intended for distributed memory systems with heterogeneous architecture, that is, programs executed on several CPUs in conjunction with co-processors, in this case Nvidia GPUs.
The software is written in \CC{} and CUDA and uses the Message Passing Interface (MPI) library~\cite{Clarke1994} to communicate between CPU processes and the CUDA API to communicate between the GPU and the CPU.

While stencil computation is a relatively simple procedure, applying linear operations to points on a grid, the complexities introduced by heterogeneity and swept time-space decomposition require a significant number of design decisions.
In this work we investigated the performance impact of the most immediately salient and configurable decisions and constrained other potential variations with reasonable or previously investigated values.
Our investigations focus on answering the following questions:
\begin{enumerate}
    \item Does the swept rule reduce time cost using the most favorable launch configurations over a large range of grid sizes?
    \item How much work should we give to the GPU in a heterogeneous system?
    \item How should we organize the stencil update formula for multi-step methods? (Further discussion in Section~\ref{sec:hPrimaryData})
    \item Does the size of the domain of dependence substantially affect performance?
\end{enumerate}

% I was thinking about pulling Program Design Features and making Primary data structure part of results.