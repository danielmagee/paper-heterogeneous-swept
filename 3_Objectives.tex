\section{Objectives} \label{sec:obj1}

This study concerns the construction and analysis of a program that applies swept time-space decomposition to explicit stencil computations intended for distributed memory systems with heterogeneous architecture, that is, computational operations are performed by several CPUs and co-processors, in particular Nvidia GPUs.
The software is written in \CC{} and CUDA and uses the Message Passing Interface (MPI) library~\cite{Clarke1994} to communicate between CPU processes and the CUDA API to communicate between the GPU and the CPU.

While stencil computation is a relatively simple procedure, applying linear operations to individual spatial
points and their neighbors and the complexities introduced both by a heterogeneous architecture and swept
time-space decomposition require a significant number of design decisions.
In this work we investigated the performance impact of the most immediately salient and configurable decisions,
and constrained other potential variations with reasonable or previously investigated values.
Our investigations focus on answering the following questions:
\begin{enumerate}
    \item Does the swept rule reduce time-cost under optimal launch bounds over the domain of grid sizes?
    \item How much work should we give to the GPU in a heterogeneous system?
    \item How should we decompose the stencils in multi-step methods? (Further discussion in Section~\ref{sec:hPrimaryData})
    \item Does the size of the domain of dependence substantially affect performance?
\end{enumerate}
